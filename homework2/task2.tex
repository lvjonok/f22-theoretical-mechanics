\section{Task 2}

\subsection{Derivation of angle of rotation around $Z$}

As $\omega_1$ and $\epsilon_1$ are given, we can convert them to $\phi(t)$:
\begin{align}
    \phi(t) = \phi_0 + \omega_1 \cdot t + \epsilon_1 \cdot \frac{t^2}{2}
\end{align}
This angle will determine the angle of rotation of cone $A$ axis around $Z$ axis.

\subsection{Calculating axes of cones}

\begin{enumerate}
    \item For simplifications in simulation, I introduced several unit vectors for initial position.
    \item Cone $A$ axis (direction from base of cone to its vertex):
          \begin{align}
              \vec{u_A} = \begin{bmatrix}
                  0                                     \\
                  -\cos{\frac{ \alpha_B - \alpha_A}{2}} \\
                  -\sin{\frac{\alpha_B - \alpha_A}{2}}
              \end{bmatrix}
          \end{align}
    \item $IC$ axis:
          \begin{enumerate}
              \item $IC$ axis is axis of intersection between cone $A$ and cone $B$ because $B$ is static and $A$ moves without slippering.
              \item All velocities of the points in $A$ can easily be found using $IC$ axis.
              \item It is kinda obvious that this axis will be rotated around $Z$ axis with angle $\phi(t)$ together will cone $A$.
              \item It is enough to calculate initial vector and then rotate it around $Z$ axis.
          \end{enumerate}
          \begin{align}
              \vec{u_{IC}} = \begin{bmatrix}
                  0                    \\
                  \sin({\alpha_B / 2}) \\
                  \cos({\alpha_B / 2})
              \end{bmatrix}
          \end{align}
    \item Radius of cone $A$ axis:
          \begin{enumerate}
              \item I needed this unit vector to calculate position of point $M$ as it is simplier to split transformation on parts and then combine them.
              \item It is simply cone $A$ axis rotated by $90\deg$ around $X$ axis.
          \end{enumerate}
          \begin{align}
              \vec{u_{rA}} = \begin{bmatrix}
                  0                     \\
                  -\sin({\alpha_A / 2}) \\
                  \cos({\alpha_A / 2})
              \end{bmatrix}
          \end{align}
\end{enumerate}

\subsection{Angular velocity of cone $A$}

\begin{enumerate}
    \item Let state radius of cone $A$:
          \begin{align}
              r_A = OM_0 \sin({\alpha_A / 2})
          \end{align}
    \item It is widely known formula for velocity: $\vec{v} = \vec{\omega} \times \vec{r}$.
    \item I discovered, that we can do reverse: $\vec{\omega} = \frac{\vec{r} \times \vec{v}}{||\vec{r}||^2}$ \href{https://www.omnicalculator.com/physics/angular-velocity}{source}.
    \item In our case it would be radius vector from  cone $A$ center axis to $IC$ axis.
          \begin{align}
              \vec{r}_{centerA} = r_A \cdot  R_z(\phi) \cdot \vec{u_{A}}            \\
              \vec{r} = \vec{r}_{centerA} - OM_0 \cdot R_z(\phi) \cdot \vec{u_{rA}} \\
              \vec{v} = \dot{\vec{r}}_{centerA}                                     \\
              \vec{\omega} = \frac{\vec{r} \times \vec{v}}{||\vec{r}||^2}
          \end{align}

\end{enumerate}

\subsection{Angular acceleration of cone $A$}

Without many words:

\begin{align}
    \vec{\epsilon} = \dot{\vec{\omega}}
\end{align}

\subsection{Position of point $M$}

\begin{enumerate}
    \item We know that point $M$ lies on the base of cone $A$
    \item That means we can find its position by first tranforming from origin to center of base of cone $A$ and then rotating it around cone $A$ axis on required angle.
    \item Firstly, I needed to find the angle on which point $M$ should be rotated.
          Imagine that circumference of cone $B$ is flattened into the line. Then rotation of cone $A$ on cone $B$ is like the wheel with radius of cone $A$ is rotating on the line.
          \begin{align}
              \theta_A = \frac{OM_0 \cdot \phi }{OM_0 \cdot \sin({\alpha_A / 2})} =\
              \frac{\phi }{\sin({\alpha_A / 2})}
          \end{align}
          This angle determines rotation of $u_{rA}$ around $u_A$.
    \item Then we can find position of point $M$:
          \begin{align}
              \vec{r}_{M} = \vec{r}_{centerA} + R_z(\phi) \cdot R_{u_A}(\theta_A) \cdot \vec{u}_{rA}
          \end{align}
\end{enumerate}

\subsection{Velocity and accelerationi of point $M$}

\begin{enumerate}
    \item Velocity of point $M$:
          \begin{align}
              \vec{v}_{M} = \dot{\vec{r}}_{M}
          \end{align}
    \item Acceleration of point $M$:
          \begin{align}
              \vec{a}_{M} = \dot{\vec{v}}_{M}
          \end{align}
    \item Tangential acceleration of point $M$:
          \begin{align}
              \vec{a}_{tM} = \vec{a}_{M} - \vec{v}_{M} \cdot \frac{\vec{v}_{M}}{||\vec{v}_{M}||}
          \end{align}

\end{enumerate}


\begin{answer}
    \subsection*{Answer:}
    \begin{enumerate}
        \item Angular velocity of cone $A$:
              \begin{align*}
                  \vec{r}_{centerA} = r_A \cdot  R_z(\phi) \cdot \vec{u_{A}}            \\
                  \vec{r} = \vec{r}_{centerA} - OM_0 \cdot R_z(\phi) \cdot \vec{u_{rA}} \\
                  \vec{v} = \dot{\vec{r}}_{centerA}                                     \\
                  \vec{\omega} = \frac{\vec{r} \times \vec{v}}{||\vec{r}||^2}
              \end{align*}
        \item Angular acceleration of cone $A$:
              \begin{align*}
                  \vec{\epsilon} = \dot{\vec{\omega}}
              \end{align*}
        \item Position of point $M$:
              \begin{align*}
                  \vec{u_{rA}} = \begin{bmatrix}
                      0                     \\
                      -\sin({\alpha_A / 2}) \\
                      \cos({\alpha_A / 2})
                  \end{bmatrix} \\
                  \vec{r}_{M} = \vec{r}_{centerA} + R_z(\phi) \cdot R_{u_A}(\theta_A) \cdot \vec{u}_{rA}
              \end{align*}
        \item Velocity of point $M$:
              \begin{align*}
                  \vec{v}_{M} = \dot{\vec{r}}_{M}
              \end{align*}
        \item Acceleration of point $M$:
              \begin{align*}
                  \vec{a}_{M} = \dot{\vec{v}}_{M}
              \end{align*}
    \end{enumerate}
\end{answer}
